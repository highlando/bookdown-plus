\documentclass[]{article}
\usepackage{lmodern}
\usepackage{amssymb,amsmath}
\usepackage{ifxetex,ifluatex}
\usepackage{fixltx2e} % provides \textsubscript
\ifnum 0\ifxetex 1\fi\ifluatex 1\fi=0 % if pdftex
  \usepackage[T1]{fontenc}
  \usepackage[utf8]{inputenc}
\else % if luatex or xelatex
  \ifxetex
    \usepackage{xltxtra,xunicode}
  \else
    \usepackage{fontspec}
  \fi
  \defaultfontfeatures{Ligatures=TeX,Scale=MatchLowercase}
\fi
% use upquote if available, for straight quotes in verbatim environments
\IfFileExists{upquote.sty}{\usepackage{upquote}}{}
% use microtype if available
\IfFileExists{microtype.sty}{%
\usepackage{microtype}
\UseMicrotypeSet[protrusion]{basicmath} % disable protrusion for tt fonts
}{}
\usepackage[a4paper]{geometry}
\usepackage[unicode=true]{hyperref}
\hypersetup{
            pdftitle={bookdown+: Authoring Articles, Mails, Guitar Chords, Chemical Molecular Formulae and Equations with R bookdown},
            pdfauthor={Peng Zhao},
            pdfborder={0 0 0},
            breaklinks=true}
\urlstyle{same}  % don't use monospace font for urls
\usepackage{natbib}
\bibliographystyle{apalike}
\usepackage{longtable,booktabs}
% Fix footnotes in tables (requires footnote package)
\IfFileExists{footnote.sty}{\usepackage{footnote}\makesavenoteenv{long table}}{}
\IfFileExists{parskip.sty}{%
\usepackage{parskip}
}{% else
\setlength{\parindent}{0pt}
\setlength{\parskip}{6pt plus 2pt minus 1pt}
}
\setlength{\emergencystretch}{3em}  % prevent overfull lines
\providecommand{\tightlist}{%
  \setlength{\itemsep}{0pt}\setlength{\parskip}{0pt}}
\setcounter{secnumdepth}{5}
% Redefines (sub)paragraphs to behave more like sections
\ifx\paragraph\undefined\else
\let\oldparagraph\paragraph
\renewcommand{\paragraph}[1]{\oldparagraph{#1}\mbox{}}
\fi
\ifx\subparagraph\undefined\else
\let\oldsubparagraph\subparagraph
\renewcommand{\subparagraph}[1]{\oldsubparagraph{#1}\mbox{}}
\fi

% set default figure placement to htbp
\makeatletter
\def\fps@figure{htbp}
\makeatother

\usepackage[boldfont,slantfont]{xeCJK}
\usepackage{multicol}

\setmainfont{Times New Roman}
\setCJKmainfont[BoldFont={SimHei},ItalicFont={KaiTi}]{SimSun}
\setsansfont{SimHei}

\setCJKfamilyfont{song}{SimSun}
\setCJKfamilyfont{kai}{KaiTi}
\setCJKfamilyfont{hei}{SimHei}
\setCJKfamilyfont{yao}{FZYaoTi}

\newcommand\song{\CJKfamily{song}}
\newcommand\kai{\CJKfamily{kai}}
\newcommand\hei{\CJKfamily{hei}}
\newcommand\yao{\CJKfamily{yao}}

\newcommand{\erhao}{\fontsize{22pt}{\baselineskip}\selectfont}
\newcommand{\xiaoerhao}{\fontsize{18pt}{\baselineskip}\selectfont}
\newcommand{\sanhao}{\fontsize{16pt}{\baselineskip}\selectfont}
\newcommand{\xiaosanhao}{\fontsize{15pt}{\baselineskip}\selectfont}
\newcommand{\sihao}{\fontsize{14pt}{\baselineskip}\selectfont}
\newcommand{\xiaosihao}{\fontsize{12pt}{\baselineskip}\selectfont}
\newcommand{\wuhao}{\fontsize{10.5pt}{\baselineskip}\selectfont}
\newcommand{\xiaowuhao}{\fontsize{9pt}{\baselineskip}\selectfont}
\newcommand{\liuhao}{\fontsize{7.5pt}{\baselineskip}\selectfont}

%%%段落首行缩进两个字
\makeatletter
\let\@afterindentfalse\@afterindenttrue
\@afterindenttrue
\makeatother
\setlength{\parindent}{2em}%中文缩进两个汉字位

%%%%%%%%%% 定理类环境的定义 %%%%%%%%%%
%% 必须在导入中文环境之后
\newtheorem{example}{例}             % 整体编号
\newtheorem{algorithm}{算法}
\newtheorem{theorem}{定理}[section]  % 按 section 编号
\newtheorem{definition}{定义}
\newtheorem{axiom}{公理}
\newtheorem{property}{性质}
\newtheorem{proposition}{命题}
\newtheorem{lemma}{引理}
\newtheorem{corollary}{推论}
\newtheorem{remark}{注解}
\newtheorem{condition}{条件}
\newtheorem{conclusion}{结论}
\newtheorem{assumption}{假设}

%%%%%%%%%% 一些重定义 %%%%%%%%%%
%% 必须在导入中文环境之后
\renewcommand{\contentsname}{目录}     % 将Contents改为目录
\renewcommand{\abstractname}{摘\ \ 要} % 将Abstract改为摘要
\renewcommand{\refname}{参考文献}      % 将References改为参考文献
\renewcommand{\indexname}{索引}
\renewcommand{\figurename}{图}
\renewcommand{\tablename}{表}
\renewcommand{\appendixname}{附录}
%\renewcommand{\proofname}{证明}
\renewcommand{\algorithm}{算法}

%%%%%%%%%%%%%%%%%%%%%%%%%%%%%%%%%%%%%%%%%%%%%%%%%%%%%%%%%%%%%%%%
%  packages
%    这部分声明需要用到的包
%%%%%%%%%%%%%%%%%%%%%%%%%%%%%%%%%%%%%%%%%%%%%%%%%%%%%%%%%%%%%%%%
\usepackage{graphicx}    % EPS 图片支持
\usepackage{indentfirst} % 中文段落首行缩进
\usepackage{bm}          % 公式中的粗体字符(用命令\boldsymbol)

%%%%%%%%%%%%%%%%%%%%%%%%%%%%%%%%%%%%%%%%%%%%%%%%%%%%%%%%%%%%%%%%
%  lengths
%    下面的命令重定义页面边距,使其符合中文刊物习惯。
%%%%%%%%%%%%%%%%%%%%%%%%%%%%%%%%%%%%%%%%%%%%%%%%%%%%%%%%%%%%%%%%
\addtolength{\topmargin}{-54pt}
\setlength{\oddsidemargin}{0.63cm}  % 3.17cm - 1 inch
\setlength{\evensidemargin}{\oddsidemargin}
\setlength{\textwidth}{14.66cm}
\setlength{\textheight}{24.00cm}    % 24.62


\title{{\hei{bookdown+: Authoring Articles, Mails, Guitar Chords, Chemical Molecular
Formulae and Equations with R bookdown}}}
\author{Peng Zhao}
\date{}

\begin{document}
\maketitle

\begin{center}
\parbox{\textwidth}{
%\rule{2em}{0pt}
\hei{摘要:}\song{
随着计算机软硬件系统日益复杂,如何保证其正确性和可靠性成为日益紧迫的问题。在为此提出的诸多理论和方法中,模型检测以其简洁明了和自动化程度高而引人注目,模型检测的研究大致涵盖以下内容:模态/时序逻辑、模型检测算法及其时空效率(特别是空间效率)的改进以及支撑工具的研制。这几个方面之间有着密切的内在联系,不同模态/时序逻辑的模型检测算法的复杂性不一样,优化算法往往是针对某些特定类型的逻辑公式,本文将就这几个方面分别加以阐述,最后介绍该领域的新进展。
}\\[5pt]

\hei{关键词:}\song{
很关键;很关键;非常关键
}
\\[5pt]
}
\end{center}

%%%%%%%%%%%%%%%%%%%%%%%%%%%%%%%%%%%%%%%%%%%%%%%%%%%%%%%%%%%%%%%%
%  英文摘要
%%%%%%%%%%%%%%%%%%%%%%%%%%%%%%%%%%%%%%%%%%%%%%%%%%%%%%%%%%%%%%%%
\begin{center}
\sihao{\textbf{
An article template for Chinese journals based on R bookdown
}}\\[7pt]
\normalsize
Dapeng Zhao~~~~~~
Xiaopeng Zhao~~~~~~
\\[7pt]
\xiaowuhao Institute of Ecology\\
University of Innsbruck, Austria\\[10pt]
\end{center}
\begin{center}
\parbox{\textwidth}{
\textbf{Abstract:} 
Model checking is an automatic technique for verifying finite-state reactive systems, such as sequential circuit designs and communication protocols. Specifications are expressed in temporal logic, and the reactive system is modeled as a state-transition graph. An efficient search procedure is used to determine whether or not the state-transition graph satisfies the specifications.
We describe the basic model checking algorithm and show how it can be used with binary decision diagrams to verify properties of large state-transition graphs. We illustrate the power of model checking to find subtle errors by verifying part of the Contingency Guidance Requirements for the Space Shuttle.\\[4pt]
\textbf{Keywords:} 
Key; Key; the Key
}

\end{center}


\chapter{Introduction}\label{introduction}

\emph{italic} \textbf{bold} \textbf{\emph{italic and bold}}
\texttt{code} R\textsuperscript{2} CO\textsubscript{2}
\href{http://pzhao.org}{my website}
\href{mailto:myemail@pzhao.org}{\nolinkurl{myemail@pzhao.org}}\footnote{footnote,
  not real} \citet{R-base} Lorem ipsum dolor sit amet, consectetur
adipiscing elit, sed do eiusmod tempor incididunt ut labore et dolore
magna aliqua. Ut enim ad minim veniam, quis nostrud exercitation ullamco
laboris nisi ut aliquip ex ea commodo consequat. Duis aute irure dolor
in reprehenderit in voluptate velit esse cillum dolore eu fugiat nulla
pariatur. Excepteur sint occaecat cupidatat non proident, sunt in culpa
qui officia deserunt mollit anim id est laborum \citet{R-bookdown}.

\begin{quote}
orem ipsum dolor sit amet, consectetur adipiscing elit, sed do eiusmod
tempor incididunt ut labore et dolore magna aliqua.
\end{quote}

\chapter{Materials and methods}\label{materials-and-methods}

\section{Sites}\label{sites}

Fig. \ref{fig:img1} is a place holder.

\begin{figure}

{\centering \includegraphics[width=0.8\linewidth]{images/img} 

}

\caption{this is the caption}\label{fig:img1}
\end{figure}

\section{Models}\label{models}

Eq. \eqref{eq:mc2} is an equation.

\begin{equation} 
E = mc^2
  \label{eq:mc2}
\end{equation}

It can be written as \(E = mc^2\).

\chapter{Results}\label{results}

Fig. \ref{fig:fig1} psum dolor sit amet, consectetur adipiscing elit,
sed do eiusmod tempor incididunt ut labore et dolore magna aliqua.

\begin{figure}

{\centering \includegraphics[width=0.8\linewidth]{article_zh_files/figure-latex/fig1-1} 

}

\caption{caption}\label{fig:fig1}
\end{figure}

Tab. \ref{tab:tab1} psum dolor sit amet, consectetur adipiscing elit,
sed do eiusmod tempor incididunt ut labore et dolore magna aliqua.

\begin{table}

\caption{\label{tab:tab1}Here is a nice table!}
\centering
\begin{tabular}[t]{rrrrl}
\toprule
Sepal.Length & Sepal.Width & Petal.Length & Petal.Width & Species\\
\midrule
5.1 & 3.5 & 1.4 & 0.2 & setosa\\
4.9 & 3.0 & 1.4 & 0.2 & setosa\\
4.7 & 3.2 & 1.3 & 0.2 & setosa\\
4.6 & 3.1 & 1.5 & 0.2 & setosa\\
5.0 & 3.6 & 1.4 & 0.2 & setosa\\
\addlinespace
5.4 & 3.9 & 1.7 & 0.4 & setosa\\
4.6 & 3.4 & 1.4 & 0.3 & setosa\\
5.0 & 3.4 & 1.5 & 0.2 & setosa\\
4.4 & 2.9 & 1.4 & 0.2 & setosa\\
4.9 & 3.1 & 1.5 & 0.1 & setosa\\
\addlinespace
5.4 & 3.7 & 1.5 & 0.2 & setosa\\
4.8 & 3.4 & 1.6 & 0.2 & setosa\\
4.8 & 3.0 & 1.4 & 0.1 & setosa\\
4.3 & 3.0 & 1.1 & 0.1 & setosa\\
5.8 & 4.0 & 1.2 & 0.2 & setosa\\
\addlinespace
5.7 & 4.4 & 1.5 & 0.4 & setosa\\
5.4 & 3.9 & 1.3 & 0.4 & setosa\\
5.1 & 3.5 & 1.4 & 0.3 & setosa\\
5.7 & 3.8 & 1.7 & 0.3 & setosa\\
5.1 & 3.8 & 1.5 & 0.3 & setosa\\
\bottomrule
\end{tabular}
\end{table}

\chapter{Conclusions}\label{conclusions}

Lorem ipsum dolor sit amet, consectetur adipiscing elit, sed do eiusmod
tempor incididunt ut labore et dolore magna aliqua. Ut enim ad minim
veniam, quis nostrud exercitation ullamco laboris nisi ut aliquip ex ea
commodo consequat. Duis aute irure dolor in reprehenderit in voluptate
velit esse cillum dolore eu fugiat nulla pariatur. Excepteur sint
occaecat cupidatat non proident, sunt in culpa qui officia deserunt
mollit anim id est laborum

\bibliography{bib/bib}

\end{document}
