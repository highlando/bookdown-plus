\begin{center}
\parbox{\textwidth}{
%\rule{2em}{0pt}
\hei{摘要:}\song{
随着计算机软硬件系统日益复杂,如何保证其正确性和可靠性成为日益紧迫的问题。在为此提出的诸多理论和方法中,模型检测以其简洁明了和自动化程度高而引人注目,模型检测的研究大致涵盖以下内容:模态/时序逻辑、模型检测算法及其时空效率(特别是空间效率)的改进以及支撑工具的研制。这几个方面之间有着密切的内在联系,不同模态/时序逻辑的模型检测算法的复杂性不一样,优化算法往往是针对某些特定类型的逻辑公式,本文将就这几个方面分别加以阐述,最后介绍该领域的新进展。
}\\[5pt]

\hei{关键词:}\song{
很关键;很关键;非常关键
}
\\[5pt]
}
\end{center}

%%%%%%%%%%%%%%%%%%%%%%%%%%%%%%%%%%%%%%%%%%%%%%%%%%%%%%%%%%%%%%%%
%  英文摘要
%%%%%%%%%%%%%%%%%%%%%%%%%%%%%%%%%%%%%%%%%%%%%%%%%%%%%%%%%%%%%%%%
\begin{center}
\sihao{\textbf{
An article template for Chinese journals based on R bookdown
}}\\[7pt]
\normalsize
Dapeng Zhao~~~~~~
Xiaopeng Zhao~~~~~~
\\[7pt]
\xiaowuhao Institute of Ecology\\
University of Innsbruck, Austria\\[10pt]
\end{center}
\begin{center}
\parbox{\textwidth}{
\textbf{Abstract:} 
Model checking is an automatic technique for verifying finite-state reactive systems, such as sequential circuit designs and communication protocols. Specifications are expressed in temporal logic, and the reactive system is modeled as a state-transition graph. An efficient search procedure is used to determine whether or not the state-transition graph satisfies the specifications.
We describe the basic model checking algorithm and show how it can be used with binary decision diagrams to verify properties of large state-transition graphs. We illustrate the power of model checking to find subtle errors by verifying part of the Contingency Guidance Requirements for the Space Shuttle.\\[4pt]
\textbf{Keywords:} 
Key; Key; the Key
}

\end{center}
